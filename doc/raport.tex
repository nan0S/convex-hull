\documentclass[11pt]{article}

\usepackage[a4paper, margin=2cm]{geometry}
\usepackage[OT4, plmath]{polski}
\usepackage{listings}
\usepackage{graphicx}
\usepackage{hyperref}
\usepackage{float}

\title{
	Obliczenia równoległe na kartach graficznych CUDA \\
	Projekt \\
	Obliczanie otoczki wypukłej \\
}
\author{Hubert Obrzut}
\date{}

\begin{document}

	\maketitle
	
	\large
	
	\textbf{Projekt}: obliczanie otoczki wypukłej na GPU.
	
	\begin{itemize}
		\item Celem projektu jest zaimplementowanie algorytmu obliczającego otoczkę wypukłą na GPU oraz porównanie jego wydajności zarówno z odpowiednikiem (algorytmu) na CPU, jak i alternatywami.
		\item Opieram się na pracy naukowej \href{https://www.researchgate.net/publication/271146554_A_Novel_Implementation_of_QuickHull_Algorithm_on_the_GPU}{\textit{A Novel Implementation of QuickHull Algorithm on the GPU}}.
		\item Praca dotyczy algorytmu \texttt{QuickHull} -- typu \textbf{Divide-and-Conquer}, podobny do algorytmu \texttt{QuickSort}.
		\item Algorytm w całości powinien działać na GPU.
		\item Do implementacji użyję głównie wykorzystać bibliotekę \texttt{Thrust} -- zarówno uproszczenie implementacji, jak i zwiększenie czytelności.
		\item Wydajność implementacji na GPU porównam z implementacją \texttt{QuickHull} na CPU. Dodatkowo zamierzam zaimplementować algorytm \texttt{Graham-Scan} na CPU (aktualna otoczka jako stos, sortowanie) jako referencja do alternatywnej implementacji obliczającej otoczkę wypukłą.
		\item Zarówno wszystkie punkty jak i otoczkę wyświetlę używając OpenGL (być może również krawędzie łączące punkty otoczki, tworzące wielokąt wypukły).
		\item Punkty do obliczenia otoczki wylosuję. Rozważam również wcześniej ustalone zbiory punktów, np. zbiory nielosowe, punkty modelu.
	\end{itemize}
	
\end{document}
